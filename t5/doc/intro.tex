\section{Introduction}
\label{sec:introduction}

In this laboratory assignment, we are asked to create a Band Pass Filter (BPF), in which we are supposed to design its architecture using one 741 OPAMP, resistors and capacitors. The goal is to achieve the best ratio between the cost of the circuit and its quality in terms of the specifications asked (Merit Figure): a central frequency $f_c=1 kHz$ and a central gain of $40 dB$. \par
To build the circuit, we used 4 resistors, 2 capacitors and a 741 OPAMP. A representative image of the circuit can be seen in \ref{fig:t5}. \par


In Section \ref{sec:analysis}, a theoretical analysis of the circuit using Octave is presented. In Section \ref{sec:simulation}, the circuit is analysed by simulation with Ngspice software and the results are compared to the theoretical results in Section \ref{sec:sbs}. Some comments about the in-person simulation are stated in Section \ref{sec:irl}. Conclusions of this study can be found in Section \ref{sec:conclusion}.

The components used have the following values:
\begin{table}[h]
\centering
\begin{small}
\caption{Values in SI units.}
\begin{tabular}{|c|c|}
\hline
$R_1$ & $1000$ \\
$R_2$ & $1000$\\
$R_3$ & $315000$ \\
$R_4$ & $1000$ \\
$C_1$ & $220\times 10^{-9}$ \\
$C_2$ & $220\times 10^{-9}$ \\
\hline
\end{tabular}
\end{small}
\end{table}

R3 e resistencias 3 de 100000 em serie + 1 de 10000 em serie + em serie com duas de 10000 em paralelo


\begin{figure}[htp] \centering
\includegraphics[width=0.8\linewidth]{t5.pdf}
\caption{Circuit in study.}
\label{fig:t5}
\end{figure}
\FloatBarrier

