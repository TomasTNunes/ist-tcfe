\section{Theoretical Analysis}
\label{sec:analysis}

In this section, the circuit shown in Figure~\ref{fig:t1} is analysed
theoretically, in terms of each branch's current and voltage.
The current flows considered in the analysis are the following:
$R_1$ from node 1 to 2; $R_2$ from node 2 to 3; $R_3$ from node 2 to 4; $R_4$ from node 4 to 0; $R_5$ from node 4 to 5; $R_6$ from node 0 to 6; $R_7$ from node 6 to 7. Also, the potencial in Node 0 was considered to be 0V (ground).

\subsection{Nodal Analysis}

This method is based on Kirchhoff's Current Law (KCL), and firstly consists in
deriving equations for the current flow in nodes not connected to voltage sources,
followed by writing aditional equations in nodes related to voltage sources. The node 
identification can be seen on Figure~\ref{fig:t1}. Moreover, the equations relative to
nodes 0, 2, 5 and 0, respectively, are

\begin{equation}
  (V_0 - V_4)G_4 + (V_0 - V_6)G_6 + (V_1 - V_2)G_1 = 0 
  \label{eq:kvl}
\end{equation}

\begin{equation}
  (V_2 - V_1)G_1 + (V_2 - V_4)G_3 + (V_2 - V_3)G_2 = 0 
  \label{eq:kvl2}
\end{equation}

\begin{equation}
  (V_5 - V_4)G_5 + I_b - I_d = 0 , I_b = K_b(V_2 - V_4) 
  \label{eq:kvl3}
\end{equation}

\begin{equation}
  (V_6 - V_7)G_7 + (V_0 - V_4)G_4 + (V_1 - V_2)G_1 = 0 
  \label{eq:kvl4}
\end{equation}

The aditional equations for the method are

\begin{equation}
  V_0 = 0
  \label{eq:kvl5}
\end{equation}

\begin{equation}
  V_1 = V_a
  \label{eq:kvl6}
\end{equation}

\begin{equation}
  V_4 - V_7 = K_c(V_0 - V_6)G_6
  \label{eq:kvl7}
\end{equation}

\begin{equation}
  I_b = (V_3 - V_2)G_2 , I_b = K_b(V_2 - V_4)
  \label{eq:kvl8}
\end{equation}

\begin{table}[h]
  \centering
  \begin{tabular}{|l|r|}
    \hline    
    {\bf Name} & {\bf Value [mA or V]} \\ \hline
    \input{../mat/Nodal_tab}
  \end{tabular}
  \caption{Nodal Analysis' variable values, where $I_j$ is expressed in milliampere and $V_j$ is expressed in Volt.}
  \label{tab:Nodal}
\end{table}
\FloatBarrier

\subsection{Mesh Analysis}

This method is based on Kirchoff's Voltage Law (KVL), and consists on analysing the circuit's meshes
after creating arbitrary currents with arbitrary flows for each mesh in the circuit. The top left mesh
was named mesh A with clockwhise current flow $I_A$; the top right mesh was named mesh B with counterclockwhise
current flow $I_B$; the bottom left mesh was named mesh C with counterclockwhise current flow $I_C$; the bottom right
mesh was named mesh D with counterclockwhise current flow $I_D$. Furthermore, the equations associated with these meshes
and currents are

\begin{equation}
  R_1 I_A + R_3(I_A + I_B) + R_4(I_A + I_C) - V_a = 0
  \label{eq:kvl}
\end{equation}

\begin{equation}
R_6 I_C + R_7 I_C - K_c I_C + R_4(I_A + I_C) = 0
  \label{eq:kvl2}
\end{equation}

\begin{equation}
  I_D = I_d
  \label{eq:kvl3}
\end{equation}

\begin{equation}
  I_B = K_b R_3(I_A + I_B)
  \label{eq:kvl4}
\end{equation}

\begin{table}[h]
  \centering
  \begin{tabular}{|l|r|}
    \hline    
    {\bf Name} & {\bf Value [mA or V]} \\ \hline
    \input{../mat/Mesh_tab}
  \end{tabular}
  \caption{Mesh Analysis' variable values, where $I_j$ is expressed in milliampere and $V_j$ is expressed in Volt.}
  \label{tab:Mesh}
\end{table}
\FloatBarrier

