\section{Introduction}
\label{sec:introduction}

The objective of this laboratory assignment is to study a circuit containing an
independent voltage source $v_s$(t), a capacitor C, seven resistors $R_1$ to $R_7$, a linearly voltage dependent current source $I_b$ and a linearly current dependent voltage source $V_d$.
The circuit can be seen in Figure \ref{fig:t2}. The study is done in different time intervals, obtaining both the natural and the forced responses of the circuit over time making use of a certain given boundary condition, as well as plotting its response to a certain stimulus for a better understanding of the circuit's behavior. The nodal analysis method is used to obtain the circuit's parameters at certain time periods (operating points) which are the basis for further analysis. \par


In Section \ref{sec:analysis}, a theoretical analysis of the circuit using Octave is
presented. In Section \ref{sec:simulation}, the circuit is analysed by
simulation with Ngspice software and the results are compared to the theoretical results in Section \ref{sec:comparison}. Conclusions of this study can be found in
Section \ref{sec:conclusion}.

\begin{figure}[htp] \centering
\includegraphics[width=0.8\linewidth]{t2.pdf}
\caption{Circuit in study.}
\label{fig:t2}
\end{figure}
\FloatBarrier

